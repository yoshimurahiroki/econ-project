\documentclass[12pt]{ectaart} % Econometrica
% --- Draft(日)では LuaLaTeX 使用を想定 ---
\usepackage{ifluatex}
\ifluatex
  \usepackage{luatexja-fontspec}
  \setmainfont{TeX Gyre Termes}
  \setsansfont{TeX Gyre Heros}
  \setmonofont{Inconsolata}
  \setmainjfont{Noto Serif CJK JP}
\fi

\usepackage{amsmath,amssymb,amsthm,mathtools}
\usepackage{graphicx,booktabs,siunitx}
\usepackage[hidelinks]{hyperref}
\usepackage{natbib}

\title{研究タイトル(日本語ドラフト可)}
\author{Your Name}
\date{\today}

\begin{document}
\begin{abstract}
要旨をここに記述します。ドラフト段階では日本語での記述も可能です。
\end{abstract}
\maketitle

\section{導入}
本研究では、経済学における重要な問題について分析を行います。日本語の混在も可能です(ドラフト段階)。

\section{理論モデル}
理論的フレームワークを記述します。

\begin{equation}
Y_i = \beta_0 + \beta_1 X_i + \epsilon_i
\end{equation}

\section{実証分析}
実証結果を記述します。

\section{結論}
研究の結果と含意を述べます。

\bibliographystyle{econometrica}
\bibliography{refs}
\end{document}
