\documentclass[aspectratio=169]{beamer}
\usetheme{metropolis} % texlive-fullに含まれる
\usepackage{ifluatex}
\ifluatex
  \usepackage{luatexja-fontspec}
  \setmainjfont{Noto Sans CJK JP}
\fi
\usepackage{amsmath,amssymb,mathtools,graphicx,booktabs}
\usepackage{minted} % コード表示(-shell-escape が必要)
\usepackage{natbib}

\title{研究発表タイトル}
\author{Your Name}
\institute{所属機関}
\date{\today}

\begin{document}
\maketitle

\begin{frame}{概要}
\begin{itemize}
  \item 研究の背景と目的
  \item 使用するデータと手法
  \item 主要な発見
  \item 政策含意
\end{itemize}
\end{frame}

\begin{frame}{研究の背景}
経済学における重要な問題について説明します。
\end{frame}

\begin{frame}[fragile]{Pythonコード例}
\begin{minted}{python}
import numpy as np
import pandas as pd
import statsmodels.api as sm

# データの読み込み
data = pd.read_csv('data/sample.csv')
print(data.describe())
\end{minted}
\end{frame}

\begin{frame}{実証結果}
回帰分析の結果を表で示します。
\end{frame}

\begin{frame}{結論}
\begin{itemize}
  \item 主要な発見
  \item 政策含意
  \item 今後の研究課題
\end{itemize}
\end{frame}

\begin{frame}[allowframebreaks]{参考文献}
\bibliographystyle{econometrica}
\bibliography{refs}
\end{frame}
\end{document}
